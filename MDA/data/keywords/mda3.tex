\documentclass{beamer}
% \documentclass[handout]{beamer}

\usetheme{dcc}

\usepackage[utf8]{inputenc}
\usepackage[american]{babel}
\usepackage{url}

\newcommand{\ie}{i.\,e.}
\newcommand{\Ie}{I.\,e.}
\newcommand{\eg}{e.\,g.}
\newcommand{\Eg}{E.\,g.}

\newcommand{\pub}{\hskip2pt\raisebox{-3.5pt}{\pgfuseimage{beamericonarticle}}\hskip1pt}

\title{Mineração de Dados Aplicada}
\subtitle{Simple but Powerful Text-Processing Commands}

\author{Loïc Cerf}

\date{August, 21$^{\text{st}}$ 2019}

\institute
{
  \scriptsize
  DCC -- ICEx -- UFMG
  \bigskip\\
  \includegraphics[width=.334113288\linewidth]{images/dcc}\hfill\includegraphics[width=.5\linewidth]{images/ufmg}
}

\subject{Simple but Powerful Text-Processing Commands}

\keywords{text files, commands, POSIX}

\hypersetup{pdfpagemode=FullScreen}

\AtBeginSection[]
{
  \begin{frame}<beamer>
    \frametitle{Outline}
    \tableofcontents[currentsection]
  \end{frame}
}

\begin{document}

\begin{frame}[plain]
  \titlepage
\end{frame}

\section*{Simple but powerful text-processing commands}

\begin{frame}
  \frametitle{Part of the Unix philosophy}
  \begin{block}{Part of the Unix philosophy}
    Doug McIlroy (inventor of Unix pipes). In \emph{A Quarter-Century
      of Unix} (1994):\\
    \begin{quotation}
      Write programs that do one thing and do it well. Write programs
      to work together. Write programs to handle text streams, because
      that is a universal interface.
    \end{quotation}
  \end{block}
\end{frame}

\begin{frame}
  \frametitle{Utilities from the 70s}
  The Unix operating system came with specific text-processing
  commands.  They are still very useful today. Since 1984, the GNU
  project has improved them a lot (new options, improved
  efficiency). The original commands are part of POSIX standards.

  \vfill
  \pause

  Most Free operating systems are POSIX-compliant: GNU/Linux, BSD,
  illumos, Haiku, etc. Mac OS X is too. Windows is not but Cygwin is a
  good compatibility layer.
\end{frame}

\begin{frame}
  \frametitle{The shell}
  The \emph{shell} interprets every command fired in a terminal or in
  an executable file, whose first line indicates the shell to use:
  \texttt{\#!/bin/sh} (any POSIX-compliant shell) or
  \texttt{\#!/bin/bash} or \texttt{\#!/bin/dash} or
  \texttt{\#!/bin/zsh}, etc.

  \vfill

  \begin{exampleblock}{A 2-line shell script}
    \texttt{\#!/bin/sh}\\
    ~\\
    \texttt{\# Get started with the MDA exercises}\\
    \texttt{wget \url{dcc.ufmg.br/~lcerf/data.tar.xz} -O - | tar -xJ}\\
    \texttt{mv data data-\$(date +\%m\%d) 2> /dev/null}
  \end{exampleblock}
\end{frame}

\begin{frame}
  \frametitle{Geeks and repetitive tasks}
  \includegraphics[width=\linewidth]{img/repetitive-tasks}
\end{frame}

\begin{frame}
  \frametitle{Standard I/O}
  POSIX text processing commands:
  \begin{itemize}
  \item process the input stream of text line by line;
  \item by default:
    \begin{itemize}
    \item read from the standard input (the keyboard if not
      redirected);
    \item write to the standard output (the terminal if not
      redirected).
    \end{itemize}
  \end{itemize}

  \vfill
  \pause

  \texttt{<}/\texttt{>} redirects the standard I/O from/to a file.  A
  pipe binds an output stream to an input stream.  It can bear a name
  (in argument of \texttt{mkfifo}) but most workflows only need the
  unnamed pipe, \texttt{|}.  \texttt{|} redirects the standard output
  of the command on the left to the standard input of the command on
  the right.
\end{frame}

\begin{frame}
  \frametitle{Getting the data}
  \texttt{\$ wget \url{dcc.ufmg.br/~lcerf/data.tar.xz} -O - | tar -xJ}

  \vfill
  \pause

  \texttt{wget} and \texttt{tar} are two GNU commands. Like all GNU
  commands:
  \begin{itemize}[<+->]
  \item the \texttt{man} command (\eg, \texttt{man wget}) gives their
    specifications;
  \item the \texttt{info} command (\eg, \texttt{info wget}) often
    provides more detailed explanations, examples of use, etc.;
  \item long options are prefixed with \texttt{{-}{-}}, short (\ie,
    one letter) options with \texttt{-} and can be grouped (\eg,
    \texttt{-xJ} equates to \texttt{-x -J});
  \item options can take (right after) an argument; \texttt{-} means
    the standard input (\texttt{/dev/stdin}) or the standard output
    (\texttt{/dev/stdout}).
  \end{itemize}
\end{frame}

\begin{frame}
  \frametitle{Reading a large text file}
  Your favorite text editor (Vim or Emacs?) loads the entire file in
  main memory, a problem if it weights gigabytes or more.

  \vfill
  \pause

  The solution is named \texttt{less}. It is the viewer for
  \texttt{man} pages.

  \vfill

  A few commands inside \texttt{less}: Page-up/down, R (repaint), F
  (follow), [0-9]+ (scroll that many lines), / (search forwards for a
  regexp), ? (search backwards for a regexp), q (quit).

  \vfill
  \pause

  \begin{exampleblock}{Exercise}
    Find, with \texttt{less}, the IP address of the first Brazilian
    visitor after the $100^{\text{th}}$ line of
    \texttt{DistroWatch/20100428/debian}.
  \end{exampleblock}
\end{frame}

\begin{frame}
  \frametitle{Outputting the first/last lines}
  Do not test your scripts on the whole dataset!

  \vfill

  \texttt{head} outputs the first lines of the input; \texttt{tail}
  its last lines.

  \vfill

  A few options: -[0-9]+ tunes the number of lines (10 by default), -n
  too but a \texttt{-}/\texttt{+} prefix asks
  \texttt{head}/\texttt{tail} to output all lines except the provided
  number of last/first lines.

  \vfill
  \pause

  \begin{exampleblock}{Exercise}
    Print the lines 5 to 15 of one file in \texttt{DistroWatch.com}'s
    logs.
  \end{exampleblock}
\end{frame}

\begin{frame}
  \frametitle{Shuffling lines}
  The \texttt{shuf} command shuffles the input lines.

  \vfill

  A few options: -n [0-9]+ outputs only the number of lines in
  argument (uniform random sampling \emph{without} replacement), -r
  allows repetitions (uniform random sampling \emph{with}
  replacement), -e specifies a line (rather than taking those in an
  input file).

  \vfill
  \pause

  \begin{exampleblock}{Exercise}
    Sample, in a uniformly random way, 100 visits to the Ubuntu page
    on April, 28$^{\text{th}}$ 2010.
  \end{exampleblock}
\end{frame}

\begin{frame}
  \frametitle{Concatenating files}
  The \texttt{cat} command concatenates files, its arguments.

  \vfill
  \pause

  \begin{exampleblock}{Exercise}
    Concatenate the files related to the visits to the Ubuntu page.
  \end{exampleblock}
\end{frame}

\begin{frame}
  \frametitle{Replacing characters}
  Given, in argument, two lists of characters, the \texttt{tr} command
  outputs its standard input after replacing (aka translating) every
  character in the first list with the one at the same position in the
  second list.  The last character in the second list is considered
  repeated up to the size of the first list.  \texttt{-} between two
  characters defines an interval.

  \vfill

  A few options: -c specifies the first list as the complement of the
  provided characters, -d deletes (rather than replaces) the
  characters in the first list, given consecutive characters found in
  the first list, -s substitutes the first one and deletes the rest
  (squeeze).
\end{frame}

\begin{frame}
  \frametitle{Basic reformatting with \texttt{tr}}
  \begin{exampleblock}{Exercise}
    Choose one of the files in \texttt{DistroWatch.com}'s logs and:
    \begin{enumerate}
    \item change its delimiters into spaces;
    \item make its country codes lower cased.
    \end{enumerate}
  \end{exampleblock}

  \vfill

  Notice that the shell processes the command line before executing
  it: the command line is broken w.r.t. whitespaces, \texttt{\$var} is
  replaced by the content of the shell variable \texttt{var},
  \texttt{*} is replaced by any string to match existing file paths,
  etc. To preserve the literal meaning:
  \begin{itemize}
  \item a backslash can precede every special character;
  \item a (portion of an) argument can be enclosed with single quotes
    (with double quotes to still let the shell interpret \texttt{\$}
    and \texttt{`}).
  \end{itemize}
\end{frame}

\begin{frame}
  \frametitle{Counting lines, words and/or characters}
  The \texttt{wc} command counts the number of input lines (option
  -l), words (option -w) and/or characters (option -m).

  \vfill
  \pause

  \begin{exampleblock}{Exercise}
    In \texttt{DistroWatch.com}'s logs, what are the numbers of visits
    per day to the Ubuntu page. Is there something special (compare to
    other distributions)?
  \end{exampleblock}
\end{frame}

\begin{frame}
  \frametitle{Selecting fields}
  The \texttt{cut} command selects fields. \texttt{cut} considers that
  there is an empty field between two subsequent delimiters.

  \vfill

  -d specifies the delimiter (tab by default), -f specifies the fields
  \texttt{cut} must keep (comma-separated numbers or intervals, using
  \texttt{-}).

  \vfill
  \pause

  \begin{exampleblock}{Exercise}
    From one of the file in \texttt{DistroWatch.com}'s logs, get rid
    of the IP addresses.
  \end{exampleblock}
\end{frame}

\begin{frame}
  \frametitle{Pasting}
  The \texttt{paste} command concatenates the lines of the input
  files.

  \vfill

  -d specifies the delimiter (tab by default).
\end{frame}

\begin{frame}
  \frametitle{Comparing}
  \texttt{comm} compares the lines of two \emph{sorted} files.  It
  outputs the lines only in the first file (first column), only in the
  second file (second column), and those in both files (third column).

  \vfill

  -1, -2 and -3 respectively remove the first, second and third column
  from the output.
\end{frame}

\begin{frame}
  \frametitle{Joining}
  \texttt{join} joins the lines of two files. The join fields of both
  files must be \emph{sorted}.

  \vfill

  A few options: -1 NUM sets the join field of the first file, -2 NUM
  for the second file, -j for both, -a \{1, 2\} outputs unpairable
  lines from the specified file \emph{too}, -v \{1, 2\} \emph{only}
  outputs unpairable lines from the specified file, -i ignores case,
  -t CHAR sets the delimiter.
\end{frame}

\begin{frame}
  \frametitle{Sorting}
  The \texttt{sort} command sorts the lines.  The \emph{locale}
  settings affect the ordering (you may want LC\_ALL=C).

  \vfill

  A few options: -r reverses the ordering, -f ignores case, -n
  numerically sorts, -c checks if sorted, -k POS1[,POS2] sorts
  according to the fields in the interval, -t sets the field
  delimiter, -u removes duplicates, -m merges already sorted files.

  \vfill
  \pause

  \begin{exampleblock}{Exercise}
    In \texttt{DistroWatch.com}'s logs:
    \begin{enumerate}
    \item How many different countries are there?
    \item On the 29th, what are the countries who originated visits to
      the site but no visit to Ubuntu's page?
    \end{enumerate}
  \end{exampleblock}
\end{frame}

\begin{frame}
  \frametitle{Reporting or omitting repeated lines}
  The \texttt{uniq} command removes \emph{adjacent} repeated lines.

  \vfill

  A few options: -c counts and reports how many repetitions, -d only
  outputs repeated lines, -u only outputs unique lines, -i ignores
  case, -f [0-9]+ ignores the first fields (number in argument).

  \vfill
  \pause

  \begin{exampleblock}{Exercise}
    In \texttt{DistroWatch.com}'s logs, what are the ten countries
    that originated the highest numbers of accesses to the index page,
    during the three days.
  \end{exampleblock}
\end{frame}

\section*{Homework}

\begin{frame}
  \frametitle{Reading for next Wednesday}
  % \begin{alertblock}{Exercise for next Wednesday}
  %   Given a list of keywords and a text, use the commands presented so
  %   far to:
  %   \begin{enumerate}
  %   \item list each keyword occurring in the text;
  %   \item count the occurrences of each keyword in the text.
  %   \end{enumerate}
  % \end{alertblock}

  % \vfill

  \begin{alertblock}{Reading for next Wednesday}
    \href{https://pt.wikipedia.org/wiki/Express\%C3\%A3o\_regular\#BRE:_express.C3.B5es\_regulares\_b.C3.A1sicas}{\emph{Basic
        Regular Expressions} in Wikipedia.}
  \end{alertblock}
\end{frame}

\section*{}

\begin{frame}
  \frametitle{License}
  \begin{block}{\copyright 2012--2019 Loïc Cerf}
    \includegraphics[height=1.5ex]{img/licenses/cc-by-sa}\hskip1pt
    These slides are licensed under the Creative Commons
    Attribution-ShareAlike 4.0 International License.
  \end{block}
\end{frame}

\end{document}